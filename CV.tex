%-------------------------------------------------------------------------------
%                             ADDITIONAL PACKAGES
%-------------------------------------------------------------------------------
\documentclass[
	a4paper,
	% showframes,
	% vline=2.2em,
	% maincolor=cvgreen,
	% sidecolor=gray!50,
	% sectioncolor=red,
	% subsectioncolor=orange,
	% itemtextcolor=black!80,
	% sidebarwidth=0.4\paperwidth,
	% topbottommargin=0.03\paperheight,
	% leftrightmargin=20pt,
	% profilepicsize=4.5cm,
	% profilepicborderwidth=3.5pt,
	% profilepicstyle=profilecircle,
	% profilepiczoom=1.0,
	% profilepicxshift=0mm,
	% profilepicyshift=0mm,
	% profilepicrounding=1.0cm,
	% logowidth=4.5cm,
	% logospace=5mm,
	% logoposition=before,
]{fortysecondscv}


% improve word spacing and hyphenation
\usepackage{microtype}
\usepackage{ragged2e}

% uncomment in case you don't want any hyphenation
% \usepackage[none]{hyphenat}

% take care of proper font encoding
\ifxetexorluatex
	\usepackage{fontspec}
	\defaultfontfeatures{Ligatures=TeX}
	%	\newfontfamily\headingfont[Path = fonts/]{segoeuib.ttf} % local font
\else
	\usepackage[utf8]{inputenc}
	\usepackage[T1]{fontenc}
	%	\usepackage[sfdefault]{noto} % use noto google font
\fi

% enable mathematical syntax for some symbols like \varnothing
\usepackage{amssymb}


\usepackage{datenumber}
\usepackage{calc}

\newcounter{datetoday}
\newcounter{diffyears}
\newcounter{diffmonths}
\newcounter{diffdays}


% \newcommand{\difftoday}[3]{%
\newcommand{\difftoday}[6]{%
      % \setmydatenumber{datetoday}{\the\year}{\the\month}{\the\day}%
      \setmydatenumber{datetoday}{#1}{#2}{#3}%
      \setmydatenumber{diffdays}{#4}{#5}{#6}%
      \addtocounter{diffdays}{-\thedatetoday}%
      \ifnum\value{diffdays}>0
        % \def\diffbefore{in }%
        \def\diffbefore{}%
        \def\diffafter{}%
      \else
        \def\diffbefore{}%
        % \def\diffafter{ago}%
        \def\diffafter{}%
        \setcounter{diffdays}{-\value{diffdays}}%
      \fi
      \setcounter{diffyears}{\value{diffdays}/365}%
      \setcounter{diffdays}{\value{diffdays}-365*\value{diffyears}}%
      \setcounter{diffmonths}{\value{diffdays}/30}%
      \setcounter{diffdays}{\value{diffdays}-30*\value{diffmonths}}%
      %
      \diffbefore
      \ifnum\value{diffyears}=0
      \else
        \ifnum\value{diffyears}>1
            \thediffyears\space years,
        \else
            \thediffyears\space year,
        \fi
      \fi
      \ifnum\value{diffmonths}=0
      \else
        \ifnum\value{diffmonths}>1
            \textit{\thediffmonths\space months}
        \else
            \textit{\thediffmonths\space month}
        \fi
      \fi
      \diffafter
}

%-------------------------------------------------------------------------------
%                            PERSONAL INFORMATION
%-------------------------------------------------------------------------------
%% mandatory information
\cvname{Alexander Smirnov}
% \cvjobtitle{Panda Scientist,\\[0.2em] Panda of the Year}

%% optional information
\cvprofilepic{pics/profile.jpg}
% \cvlogopic{pics/profile.jpg}

\cvbirthday{May 21, 1999}
\cvaddress{Russia, Saint Petersburg} % Use \newline if more than 1 line is required
\cvcustomdata{\faGithub}{\href{https://github.com/SmirnovAlexander}{SmirnovAlexander}}
\cvmail{ru.alexander.smirnov@gmail.com}
\cvcustomdata{\faTelegram}{\href{https://t.me/furiousteabag}{@furiousteabag}}
\cvphone{+79119727982}

% personal website
% \cvsite{https://pandascience.net}

% pgp key
% \cvkey{4096R/FF00FF00}{0xAABBCCDDFF00FF00}

%-------------------------------------------------------------------------------
%                              SIDEBAR 1st PAGE
%-------------------------------------------------------------------------------
% add more profile sections to sidebar on first page
\addtofrontsidebar{
	% include gosquare national flags from https://github.com/gosquared/flags;
	% naming according to ISO 3166-1 alpha-2 country codes
	\graphicspath{{pics/flags/}}

	% % social network accounts incl. proper hyperlinks
	% \profilesection{Social Network}
	% 	\begin{icontable}{2.5em}{1em}
	% 		\social{\aiOverleafSquare}
	% 			{https://de.overleaf.com/latex/templates/forty-seconds-cv/pztcktmyngsk}
	% 			{Overleaf Template Link}
	% 		\social{\faGithub}
	% 			{https://github.com/PandaScience/FortySecondsCV}
	% 			{Github Project Page Link}
	% 	\end{icontable}

	\profilesection{Languages}
	\pointskill{\flag{RU.png}}{Russian}{5}
	\pointskill{\flag{GB.png}}{English}{4}

}

%-------------------------------------------------------------------------------
%                              SIDEBAR 2nd PAGE
%-------------------------------------------------------------------------------
\addtobacksidebar{
	\profilesection{Skills}
	\pointskill{\faLinux}{Linux}{3}[4]
	\skill[1.8em]{\faCaretRight}{arch, i3}
	\skill[1.8em]{\faCaretRight}{vim, zsh}
	\pointskill{\faPython}{Python}{3}[4]
	\skill[1.8em]{\faCaretRight}{scikit-learn, matplotlib}
	\skill[1.8em]{\faCaretRight}{nltk, gensim}
	\skill[1.8em]{\faCaretRight}{tensorflow}
	\skill[1.8em]{\faCaretRight}{surprise}
	\skill[1.8em]{\faCaretRight}{flask}
	\skill[1.8em]{\faCaretRight}{jupyter notebook}
	\pointskill{\faThumbsUp}{Recommender Systems}{3}[4]
	\skill[1.8em]{\faCaretRight}{collaborative filtering}
	\skill[1.8em]{\faCaretRight}{content-based filtering}
	\skill[1.8em]{\faCaretRight}{session filtering}
	\skill[1.8em]{\faCaretRight}{cold start}
	\skill[1.8em]{\faCaretRight}{matrix factorization}
	\pointskill{\faCogs}{Machine Learning}{2}[4]
	\skill[1.8em]{\faCaretRight}{supervised learning}
	\skill[1.8em]{\faCaretRight}{clustering}
	\skill[1.8em]{\faCaretRight}{dimensionality reduction}
	\skill[1.8em]{\faCaretRight}{data visualization}
	\skill[1.8em]{\faCaretRight}{EDA}
	\pointskill{\faCode}{Development}{2}[4]
	\skill[1.8em]{\faCaretRight}{gitlab, CI / CD}
	\skill[1.8em]{\faCaretRight}{k8s}
	\skill[1.8em]{\faCaretRight}{microservices}
	\pointskill{\faDatabase}{DBMS}{2}[4]
	\skill[1.8em]{\faCaretRight}{MySQL}
	\skill[1.8em]{\faCaretRight}{MongoDB}
	\skill[1.8em]{\faCaretRight}{ClickHouse}
	\skill[1.8em]{\faCaretRight}{Redis}
	\skill[1.8em]{\faCaretRight}{Amazon S3}
	\skill[1.8em]{\faCaretRight}{Oracle}
	\pointskill{\faChrome}{Web Development}{2}[4]
	\skill[1.8em]{\faCaretRight}{HTML, CSS, JS}
	\skill[1.8em]{\faCaretRight}{JQ, AJAX, bootstrap}
	\pointskill{\faWindows}{.NET}{2}[4]
	\skill[1.8em]{\faCaretRight}{C\#}
	\skill[1.8em]{\faCaretRight}{F\#}
	\pointskill{\faHandshake}{Multi-agent Systems}{2}[4]
	\pointskill{\faJava}{Java}{1}[4]
	\skill[1.8em]{\faCaretRight}{android development}
	\pointskill{\faCuttlefish}{C++}{1}[4]





	% \aboutme{
	% 	The giant panda is a terrestrial animal and primarily spends its life
	% 	roaming and feeding in the bamboo forests of the Qinling Mountains and in
	% 	the hilly province of Sichuan.
	% }

	% \profilesection{Diagrams}
	% \begin{sidebarminipage}
	% 	\chartlabel{Bubble}
	% 	\chartlabel{Diagrams}
	% 	\chartlabel{with}
	% 	\chartlabel{proper}
	% 	\chartlabel{overflow}
	% 	\chartlabel{protection}
	% 	\chartlabel{for}
	% 	\chartlabel{labels}
	% \end{sidebarminipage}

	% \begin{figure}\centering
	% 	\smartdiagram[bubble diagram]{
	% 		\textcolor{white}{\textbf{Being a}} \\
	% 		\textcolor{white}{\textbf{Panda}}, % center bubble
	% 		\textcolor{black!90}{Eating},
	% 		\textcolor{black!90}{Sleeping},
	% 		\textcolor{black!90}{Rolling},
	% 		\textcolor{black!90}{Playing},
	% 		\textcolor{black!90}{Chilling}
	% 	}
	% \end{figure}

	% \chartlabel{Wheel Chart}

	% \wheelchart{3.7em}{2em}{%
	% 20/3em/maincolor!50/Chill,
	% 15/3em/maincolor!15/Play,
	% 30/4em/maincolor!40/Sleep,
	% 20/3em/maincolor!20/Eat
	% }

	% \profilesection{Barskills}
	% \barskill{\faSkyatlas}{Wearing asian rice hats}{60}
	% \barskill{\faImage}{Playing Chess}{30}
	% \barskill{\faMusic}{Playing the bamboo flute}{50}

	% \profilesection{Memberships}
	% \begin{memberships}
	% 	\membership[4em]{pics/logo.png}{PandaScience.net}
	% 	\membership[4em]{pics/logo.png}{Some longer text spanning over more than
	% 		only one line}
	% \end{memberships}
}


%-------------------------------------------------------------------------------
%                         TABLE ENTRIES RIGHT COLUMN
%-------------------------------------------------------------------------------
\begin{document}

\par
\makefrontsidebar{}

\cvsection{Working Experience}
\begin{cvtable}[1.5]

    \cvitem{08.2020 -- now \\ \difftoday{\the\year}{\the\month}{\the\day}{2020}{08}{05}}
        {Recommender Systems Engineer}
        {IT Service, startup}
        {Researching, evaluating and ensembling different approaches for builiding news \& videos recommender systems. \\
         Extracting training data from logs. Inventing new logs to collect. \\
         Solving probles such as cold start, data sparcity, instant users' actions influence on further recommendations. \\
         Deploying solution to production using microservice architecture. \\
         \colorbox{cvsidecolor}{python} \colorbox{cvsidecolor}{recommender systems}
         \colorbox{cvsidecolor}{matrix factorization} 
         \colorbox{cvsidecolor}{microservices} \colorbox{cvsidecolor}{high load}
         \colorbox{cvsidecolor}{clickhouse} \colorbox{cvsidecolor}{redis} 
         \colorbox{cvsidecolor}{S3}
         \colorbox{cvsidecolor}{mySQL} \colorbox{cvsidecolor}{mongoDB} 
         \colorbox{cvsidecolor}{flask}
         \colorbox{cvsidecolor}{docker} \colorbox{cvsidecolor}{gitlab CI} 
         \colorbox{cvsidecolor}{k8s}
    }


\end{cvtable}

\cvsection{Internships}
\begin{cvtable}[1.5]

	\cvitem{09.2019 -- 12.2019 \\ \difftoday{2019}{12}{31}{2019}{09}{01}}
        {Machine Learning Intern}
        {JetBrains}
        {Predicting core properties (type, carbonate, ruin, saturation)
         using core samples images. \\
         Deep learning was used to detect properties; zones
         were highlighted using segmentaion algorithms. \\
         \colorbox{cvsidecolor}{research} 
         \colorbox{cvsidecolor}{python} \colorbox{cvsidecolor}{tensorflow}
         \colorbox{cvsidecolor}{CV} \colorbox{cvsidecolor}{segmentation}
        }
    \cvitem{07.2019 -- 08.2019 \\ \difftoday{2019}{07}{1}{2019}{08}{31}}
        {Machine Learning Intern}
        {Belkasoft}
        {Collecting data and train algorithm to find pictures that
         contain circles and arrows on them. \\
         \colorbox{cvsidecolor}{python} \colorbox{cvsidecolor}{tensorflow}
         \colorbox{cvsidecolor}{CV} \colorbox{cvsidecolor}{object detection}
         \colorbox{cvsidecolor}{C\#}
        }
\end{cvtable}


\cvsection{Education}

\cvsubsection{Study}
\begin{cvtable}[1.5]
	\cvitem{2017 -- now}
        {Bachelor Studies}
        {Saint Petersburg State University}
        {Software and Administration of Information Systems,
         Department of Information and Analytical Systems}
        \cvitem{}
            {\href{https://github.com/SmirnovAlexander/RecommenderSystemPaper/blob/master/paper/paper.pdf}
            {Bachelor's Thesis}}
            {Department of Information and Analytical Systems}
            {A hybrid approach for news recommender system using optimization methods}
        \cvitem{}
            {\href{https://github.com/SmirnovAlexander/JBCoreSamples/blob/master/papers/text/AutomaticCoreSamplesClassification\%20-\%20diploma.pdf}
            {Coursework Thesis}}
            {Department of Information and Analytical Systems}
            {Automatic core samples classification}
	\cvitem{2014 -- 2017}
        {Secondary education}{Physics and Mathematics Lyceum No. 239}
        {Best Physics and Mathematics Lyceum in Russia}
\end{cvtable}

\newpage
\makebacksidebar

\cvsection{Education}

\cvsubsection{Online courses}
\begin{cvtable}[1.5]
	\cvitem{2021 -- now}
        {Finding Structure in Data}
        {MIPT}
        {Clustering, matrix factorization, topic modelling, outliers 
         searching and visualizations}
	\cvitem{2020 -- now}
        {Introduction to Recommender Systems}
        {University of Minnesota}
        {Non-Personalized and Content-Based recommenders}
	\cvitem{2020 -- now}
        {Algorithms. Theory and Practice. Methods}
        {Computer Science Center}
        {Algorithms introduction}
	\cvitem{2020 -- now}
        {Programming in C++}
        {Computer Science Center}
        {Main concepts of C++}
    \cvitem{2021}
        {\href{https://coursera.org/share/e50e6899962fe4d392fe65f6d9a60472}
        {Training on Labeled Data}}
        {MIPT}
        {Supervised learning models: linear models, decision trees, models composition}
    \cvitem{2020}
        {\href{https://stepik.org/cert/321173}
        {Web Development for Beginners}}
        {ITC}
        {Basics of web development}
    \cvitem{2019}
        {\href{https://coursera.org/share/faf5859452fbdda8a36084effc34793a}
        {Mathematics and Python for Data Analysis}}
        {MIPT}
        {Refresh mathematics knowledge applied in data analysis and
		 review python libraries that are related}
	\cvitem{2019}
        {Introduction to TensorFlow}
        {deeplearning.ai}
        {Examples of how to work with TensorFlow library}
\end{cvtable}

\cvsubsection{Offline courses}
\begin{cvtable}[1.5]
    \cvitem{2021}
        {\href{https://vk.com/math_ai}
        {Forum on Mathematics and AI}}
        {MIPT}
        {5-day intensive forum filled with lectures on sota AI researches}
    \cvitem{2016}
        {\href{https://education.dellemc.com/content/emc/en-us/student-star.html}
        {Student STAR Program-Russia CoE track}}
        {EMC}
        {Project management course}
    \cvitem{2015 -- 2016}
        {\href{https://myitschool.ru/}
        {Samsung IT School}}
        {Samsung}
        {Android development course}
\end{cvtable}

\cvsubsection{Books}
\begin{cvtable}[1.5]
    \cvitem{2021}
        {\href{https://en.wikipedia.org/wiki/The_Lean_Startup}
        {The Lean Startup}}
        {Eric Ries}
        {Successful startup launching guideline}
    \cvitem{2020}
        {\href{http://linux-training.be/linuxfun.pdf}
        {Linux Fundamentals}}
        {Paul Cobbaut}
        {Introduction to using Linux from the command line}
\end{cvtable}

% \cvsection{Publications}
% \begin{cvtable}
% 	\cvpubitem{Cooking: 100 recipes for lazy Pandas}{Me and My Panda Friends}
% 		{Panda's Culinary World}{2010}
% 	\cvpubitem{Pandastasia}{Still Me}{Bamboo Books Assoc.}{2005}
% \end{cvtable}

% \newgeometry{
% 	top=\topbottommargin,
% 	bottom=\topbottommargin,
% 	right=\leftrightmargin,
% 	left=\leftrightmargin
% }

\addtobacksidebar{}

\newpage
\begin{sidebar}
    \nameandjob

    \setlength{\parskip}{1ex}

    % \profilesection{About Me}
	% \aboutme{
    %     \faHeart{} programming
	% }
\end{sidebar}

\cvsection{Projects}
\begin{cvtable}[1.5]
	\cvitem{2020}
        {\href{http://git.asmirnov.xyz/}
        {Git Server}}{}
        {My own private git server hosted on raspberry pi}
	\cvitem{2020}
        {\href{https://github.com/SmirnovAlexander/Tetris}
        {Tetris}}{}
        {Terminal tetris game for *nix operating system}
	\cvitem{2019}
        {\href{https://github.com/SmirnovAlexander/EmojiCommunicator}
        {Emoji Communicator}}{}
        {Visualizing text with emoji letters}
	\cvitem{2019}
        {\href{https://github.com/SmirnovAlexander/OneShotLearningSiameseNetworks}
        {Siamese Neural Networks}}{}
        {Implementation of siamese network for one-shot learning task}
	\cvitem{2019}
        {\href{https://github.com/SmirnovAlexander/WebsiteClassifier}
        {Classifying Browser Extension}}{}
        {Browser extension that allows you to classify site if it is suitable for children to watch}
	\cvitem{2018 -- 2019}
        {\href{https://github.com/SmirnovAlexander/MemDer}
        {Android Meme Application}}{}
        {An android app that show memes depending on your preferences}
	\cvitem{2018 -- 2019}
        {\href{https://github.com/SmirnovAlexander/SuperManager}
        {Custom File Manager}}{}
        {Illustrates a bunch of programming patterns and optimization techniques}
\end{cvtable}

\cvsection{Achievements}
\begin{cvtable}
	\cvitem{2019}
        {\href{https://github.com/SmirnovAlexander/EmojiCommunicator}
        {2nd place}}
        {PhotoLab PhotoHack}
        {\empty}
	\cvitem{2018}
        {\href{https://github.com/SmirnovAlexander/MemDer}
        {Finals}}
        {VK Hackathon}
        {\empty}
\end{cvtable}

\cvsection{Hackathons}
\begin{cvtable}[1.5]
    \cvitem{2021}
        {\href{https://github.com/SmirnovAlexander/QuestionsClassifier}
        {MIPT Forum Hackaton}}
        {Moscow}
        {Classifying quiz questions if they written by professionals or not}
    \cvitem{2020}
        {\href{https://github.com/SmirnovAlexander/ParseCV}
        {TrudHack 2}}
        {Saint Petersburg}
        {CV parser}
	\cvitem{2019}
        {Tender Hack.Spb}
        {Saint Petersburg}
        {Search algorithm}
    \cvitem{2019}
        {\href{https://github.com/SmirnovAlexander/EmojiCommunicator}
        {PhotoLab PhotoHack}}
        {Veliky Novgorod}
        {Program that help users to express themselves in chats}
	\cvitem{2019}
        {QuantNet}
        {Saint Petersburg}
        {Trading algorithm}
	\cvitem{2019}
        {SPbU GameJam 2019}
        {Saint Petersburg}
        {Game}
    \cvitem{2018}
        {\href{https://github.com/SmirnovAlexander/MemDer}
        {VK Hackathon}}
        {Saint Petersburg}
        {Entertaining app}
\end{cvtable}


\cvsignature

\end{document}
