%-------------------------------------------------------------------------------
%                             ADDITIONAL PACKAGES
%-------------------------------------------------------------------------------
\documentclass[
	a4paper,
	% showframes,
	% vline=2.2em,
	% maincolor=cvgreen,
	% sidecolor=gray!50,
	% sectioncolor=red,
	% subsectioncolor=orange,
	% itemtextcolor=black!80,
	% sidebarwidth=0.4\paperwidth,
	% topbottommargin=0.03\paperheight,
	% leftrightmargin=20pt,
	% profilepicsize=4.5cm,
	% profilepicborderwidth=3.5pt,
	% profilepicstyle=profilecircle,
	% profilepiczoom=1.0,
	% profilepicxshift=0mm,
	% profilepicyshift=0mm,
	% profilepicrounding=1.0cm,
	% logowidth=4.5cm,
	% logospace=5mm,
	% logoposition=before,
]{fortysecondscv}


% improve word spacing and hyphenation
\usepackage{microtype}
\usepackage{ragged2e}

% uncomment in case you don't want any hyphenation
% \usepackage[none]{hyphenat}

% take care of proper font encoding
\ifxetexorluatex
	\usepackage{fontspec}
	\defaultfontfeatures{Ligatures=TeX}
	%	\newfontfamily\headingfont[Path = fonts/]{segoeuib.ttf} % local font
\else
	\usepackage[utf8]{inputenc}
	\usepackage[T1]{fontenc}
	%	\usepackage[sfdefault]{noto} % use noto google font
\fi

% enable mathematical syntax for some symbols like \varnothing
\usepackage{amssymb}

%-------------------------------------------------------------------------------
%                            PERSONAL INFORMATION
%-------------------------------------------------------------------------------
%% mandatory information
\cvname{Alexander Smirnov}
% \cvjobtitle{Panda Scientist,\\[0.2em] Panda of the Year}

%% optional information
\cvprofilepic{pics/profile.jpg}
% \cvlogopic{pics/profile.jpg}

\cvbirthday{May 21, 1999}
\cvaddress{Russia, Saint Petersburg} % Use \newline if more than 1 line is required
\cvcustomdata{\faGithub}{\href{https://github.com/SmirnovAlexander}{SmirnovAlexander}}
\cvmail{ru.alexander.smirnov@gmail.com}
\cvcustomdata{\faTelegram}{\href{https://t.me/furiousteabag}{@furiousteabag}}
\cvphone{+79119727982}

% personal website
% \cvsite{https://pandascience.net}

% pgp key
% \cvkey{4096R/FF00FF00}{0xAABBCCDDFF00FF00}

%-------------------------------------------------------------------------------
%                              SIDEBAR 1st PAGE
%-------------------------------------------------------------------------------
% add more profile sections to sidebar on first page
\addtofrontsidebar{
	% include gosquare national flags from https://github.com/gosquared/flags;
	% naming according to ISO 3166-1 alpha-2 country codes
	\graphicspath{{pics/flags/}}

	% % social network accounts incl. proper hyperlinks
	% \profilesection{Social Network}
	% 	\begin{icontable}{2.5em}{1em}
	% 		\social{\aiOverleafSquare}
	% 			{https://de.overleaf.com/latex/templates/forty-seconds-cv/pztcktmyngsk}
	% 			{Overleaf Template Link}
	% 		\social{\faGithub}
	% 			{https://github.com/PandaScience/FortySecondsCV}
	% 			{Github Project Page Link}
	% 	\end{icontable}

	\profilesection{Languages}
	\pointskill{\flag{RU.png}}{Russian}{5}
	\pointskill{\flag{GB.png}}{English}{4}

}

%-------------------------------------------------------------------------------
%                              SIDEBAR 2nd PAGE
%-------------------------------------------------------------------------------
\addtobacksidebar{
	\profilesection{Skills}
	\pointskill{\faLinux}{Linux}{3}[4]
	\skill[1.8em]{\faCaretRight}{Arch, i3}
	\skill[1.8em]{\faCaretRight}{vim, zsh}
	\pointskill{\faPython}{Python}{3}[4]
	\skill[1.8em]{\faCaretRight}{NumPy, SciPy, Matplotlib}
	\skill[1.8em]{\faCaretRight}{Scikit-learn, Tensorflow}
	\pointskill{\faCogs}{Data science}{2}[4]
	\skill[1.8em]{\faCaretRight}{Recommender systems}
	\skill[1.8em]{\faCaretRight}{Data collection}
	\skill[1.8em]{\faCaretRight}{Data visualization}
	\skill[1.8em]{\faCaretRight}{EDA}
	\skill[1.8em]{\faCaretRight}{Performance metrics}
	\pointskill{\faChrome}{Web development}{2}[4]
	\skill[1.8em]{\faCaretRight}{HTML, CSS, JS, PHP}
	\skill[1.8em]{\faCaretRight}{JQ, AJAX, Bootstrap}
	\skill[1.8em]{\faCaretRight}{MySQL}
	\pointskill{\faJava}{Java}{2}[4]
	\skill[1.8em]{\faCaretRight}{Android development}
	\pointskill{\faWindows}{.NET}{2}[4]
	\skill[1.8em]{\faCaretRight}{C\#}
	\skill[1.8em]{\faCaretRight}{F\#}
	\pointskill{\faHandshake}{Multi-agent system}{2}[4]
	\pointskill{\faCuttlefish}{C++}{1}[4]





	% \aboutme{
	% 	The giant panda is a terrestrial animal and primarily spends its life
	% 	roaming and feeding in the bamboo forests of the Qinling Mountains and in
	% 	the hilly province of Sichuan.
	% }

	% \profilesection{Diagrams}
	% \begin{sidebarminipage}
	% 	\chartlabel{Bubble}
	% 	\chartlabel{Diagrams}
	% 	\chartlabel{with}
	% 	\chartlabel{proper}
	% 	\chartlabel{overflow}
	% 	\chartlabel{protection}
	% 	\chartlabel{for}
	% 	\chartlabel{labels}
	% \end{sidebarminipage}

	% \begin{figure}\centering
	% 	\smartdiagram[bubble diagram]{
	% 		\textcolor{white}{\textbf{Being a}} \\
	% 		\textcolor{white}{\textbf{Panda}}, % center bubble
	% 		\textcolor{black!90}{Eating},
	% 		\textcolor{black!90}{Sleeping},
	% 		\textcolor{black!90}{Rolling},
	% 		\textcolor{black!90}{Playing},
	% 		\textcolor{black!90}{Chilling}
	% 	}
	% \end{figure}

	% \chartlabel{Wheel Chart}

	% \wheelchart{3.7em}{2em}{%
	% 20/3em/maincolor!50/Chill,
	% 15/3em/maincolor!15/Play,
	% 30/4em/maincolor!40/Sleep,
	% 20/3em/maincolor!20/Eat
	% }

	% \profilesection{Barskills}
	% \barskill{\faSkyatlas}{Wearing asian rice hats}{60}
	% \barskill{\faImage}{Playing Chess}{30}
	% \barskill{\faMusic}{Playing the bamboo flute}{50}

	% \profilesection{Memberships}
	% \begin{memberships}
	% 	\membership[4em]{pics/logo.png}{PandaScience.net}
	% 	\membership[4em]{pics/logo.png}{Some longer text spanning over more than
	% 		only one line}
	% \end{memberships}
}


%-------------------------------------------------------------------------------
%                         TABLE ENTRIES RIGHT COLUMN
%-------------------------------------------------------------------------------
\begin{document}

\par
\makefrontsidebar{}

\cvsection{Working Experience}
\begin{cvtable}[1.5]


	\cvitem{08.2020 -- now}{Recommender systems engineer}{IT Service, startup}
	{Creating and evaluating news \& videos recommender systems.}

\end{cvtable}

\cvsection{Internships}
\begin{cvtable}[1.5]

	% \cvitem{currently}{}{}{}

	\cvitem{08.2019 -- 12.2019}{Machine learning intern}{JetBrains}
	{Predict core properties (type, carbonate, ruin, saturation)
		using core samples images.}

	\cvitem{07.2019 -- 08.2019}{Machine learning intern}{Belkasoft}
	{Collect data and train algorithm to find pictures that
		contain circles and arrows on them.}
\end{cvtable}


\cvsection{Education}

\cvsubsection{Study}
\begin{cvtable}[1.5]
	\cvitem{2017 -- now}{Bachelor Studies}{Saint Petersburg State University}
	{Software and Administration of Information Systems,
		Department of Information and Analytical Systems.}
	\cvitem{}{Coursework Theses}{Department of Information and Analytical Systems.}
	{Automatic core samples classification.}

	\cvitem{2014 -- 2017}{Secondary education}{Physics and Mathematics Lyceum No. 239}
	{ }
\end{cvtable}

\newpage
\makebacksidebar

\cvsection{Education}

\cvsubsection{Online courses}
\begin{cvtable}[1.5]
	\cvitem{2020 -- now}{Introduction to Recommender Systems}{University of Minnesota}
	{Non-Personalized and Content-Based recommenders.}
	\cvitem{2020 -- now}{Algorithms. Theory and practice. Methods}{Computer Science Center}
	{Algorithms introduction.}
	\cvitem{2020 -- now}{Programming in C++}{Computer Science Center}
	{Main concepts of C++.}
	\cvitem{2019 -- now}{Training on labeled data}{MIPT}
	{Supervised learning models: linear models, decision trees,
		models composition.}
	\cvitem{2020}{Linux Fundamentals}{Paul Cobbaut}
	{Introduction to using Linux from the command line.}
	\cvitem{2020}{Web development for beginners}{ITC}
	{Basics of web development.}
	\cvitem{2019}{Mathematics and python for data analysis}{MIPT}
	{Refresh mathematics knowledge applied in data analysis and
		review python libraries that are related.}
	\cvitem{2019}{Introduction to TensorFlow}{deeplearning.ai}
	{Examples of how to work with TensorFlow library.}
\end{cvtable}

\cvsubsection{Offline courses}
\begin{cvtable}[1.5]
	\cvitem{2016}{Student STAR Program-Russia CoE track}{EMC}
	{Project management course.}
	\cvitem{2015 -- 2016}{Samsung IT School}{Samsung}
	{Android development course.}
\end{cvtable}

% \cvsection{Publications}
% \begin{cvtable}
% 	\cvpubitem{Cooking: 100 recipes for lazy Pandas}{Me and My Panda Friends}
% 		{Panda's Culinary World}{2010}
% 	\cvpubitem{Pandastasia}{Still Me}{Bamboo Books Assoc.}{2005}
% \end{cvtable}

% \newgeometry{
% 	top=\topbottommargin,
% 	bottom=\topbottommargin,
% 	right=\leftrightmargin,
% 	left=\leftrightmargin
% }

\addtobacksidebar{}

\newpage
\begin{sidebar}
    \nameandjob

    \setlength{\parskip}{1ex}

    \profilesection{About Me}
	\aboutme{
        \faHeart{} programming
	}
\end{sidebar}

\cvsection{Projects}
\begin{cvtable}[1.5]
	\cvitem{2020}{\href{https://github.com/SmirnovAlexander/Tetris}{Tetris}}{}
	{Terminal tetris game for *nix operating system.}
	\cvitem{2019}{\href{https://github.com/SmirnovAlexander/EmojiCommunicator}{Emoji communicator}}{}
	{Visualizing text with emoji letters.}
	\cvitem{2019}{\href{https://github.com/SmirnovAlexander/OneShotLearningSiameseNetworks}{Siamese Neural Networks}}{}
	{Implementation of siamese network for one-shot learning task.}
	\cvitem{2019}{\href{https://github.com/SmirnovAlexander/WebsiteClassifier}{Classifying browser extension}}{}
	{Browser extension that allows you to classify site if it is suitable for children to watch.}
	\cvitem{2018 -- 2019}{\href{https://github.com/SmirnovAlexander/MemDer}{Android meme application}}{}
	{An android app that show memes depending on your preferences.}
	\cvitem{2018 -- 2019}{\href{https://github.com/SmirnovAlexander/SuperManager}{Custom file manager}}{}
	{It illustrates a bunch of programming patterns and optimization techniques.}
\end{cvtable}

\cvsection{Achievements}
\begin{cvtable}

	\cvitem{2019}{2nd place}{PhotoLab PhotoHack}
	{\empty}

	\cvitem{2018}{Finals}{VK Hackathon}
	{\empty}

\end{cvtable}

\cvsection{Hackathons}
\begin{cvtable}[1.5]

	\cvitem{2020}{TrudHack 2}{Saint Petersburg}
    {CV parser}

	\cvitem{2019}{Tender Hack.Spb}{Saint Petersburg}
	{Search algorithm}

	\cvitem{2019}{PhotoLab PhotoHack}{Veliky Novgorod}
	{Program that help users to express themselves in chats}

	\cvitem{2019}{QuantNet}{Saint Petersburg}
	{Trading algorithm}

	\cvitem{2019}{SPbU GameJam 2019}{Saint Petersburg}
	{Game}

	\cvitem{2018}{VK Hackathon}{Saint Petersburg}
	{Entertaining app}

\end{cvtable}


\cvsignature

\end{document}
